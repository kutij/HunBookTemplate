\chapter*{Előszó}
\addcontentsline{toc}{chapter}{Előszó}

A könyv megszületését az Prof. Drexler Dániel motiválta. A cél pálinkafőzésre is alkalmas rezsó kifejlesztése, egyetemi alkalmazásának bevezetése.


Mélyebb érdeklődés esetén ajánljuk mátrixanalízis témába Rózsa Pál könyvét \cite{RozsaPalMatrix}, optimalizációs területen \cite{polyak1987introduction}, Bártfai Pál a lineáris algebra és az $n$-dimenziós geometria kapcsolatának rigorózus végigvezetését bemutató könyvét \cite{BartfaiNDimLinGeom}, stb.


\chapter*{Jelölések és rövidítések jegyzéke}
\addcontentsline{toc}{chapter}{Jelölések és rövidítések jegyzéke}


\subsection*{Jelölések}


\begin{tabular}{ll}
	Skalár értékek: & $a,b,c,\dots,\alpha,\beta,\gamma,\dots$ \\
	Oszlop/sor vektorok: & $\mathbf a,\mathbf b,\mathbf c,\dots$ \\
	Mátrixok: &$\mathbf A,\mathbf B, \mathbf C,\dots$\\
	Koordinátarendszerek (keretek), pontok: &$\mathrm A,\mathrm B, \mathrm C,\dots$\\
	Az $A$ pont koordinátái az $R$ keretben: &$\mathbf a^{(R)}$\\
	Az $O$ pontból a $P$ pontba mutató vektor: &$\overrightarrow{OP}$\\
	 \ \ ennek koordinátái az $R$ keretben: &$\overrightarrow{OP}^{(R)}$\\
	 Az $\mathbf u_1$, $\mathbf u_2$, $\mathbf u_3$ mátrixok által kifeszített & \\
	 \ \ $\{ \alpha_1\mathbf u_1+\alpha_2\mathbf u_2+... | \alpha_1\in\mathbb R, \alpha_2\in\mathbb R, ... \}$ & $\Span{\mathbf u_1,\mathbf u_2,\mathbf u_3,...}$ \\
	 \ \ tér: & \\
	 Az $\mathbf M$ mátrix képtere: &$\Im{\mathbf M}$\\
	 Az $\mathbf M$ mátrix nulltere: &$\Ker{\mathbf M}$
\end{tabular}

\subsection*{Rövidítések}

\begin{tabular}{ll}
	Szabadságfok (Degree of Freedom):& $\DoF$ \\
	Redundanciafok (Degree of Redundancy):& $\DoR$ \\
	Lineáris intERPoláció: & LERP \\
	Gömbi lineáris interpoláció (Spherical LERP): & SLERP \\
	Normalizált lineáris interpoláció (Normalized LERP): & NLERP \\
	Tool Center Point & TCP
\end{tabular}

\subsection*{Vektor és mátrix műveletek}

\begin{tabular}{ll}
	Az $f$ skalár függvény gradiense:& $\grad_{f}$\\
	
	A $\mathbf v$ vektor transzponáltja:& ${\mathbf v}^{\T}$\\
	
	A $\mathbf v$ vektor 2-es normája (hossza):& $||\mathbf v||$\\
	
	A $\mathbf v$ vektor normalizáltja (iránya):& $\Normalize{\mathbf v}$\\
	
	A $\mathbf v$ és $\mathbf w$ vektorok skaláris szorzata:& ${\mathbf v}^{\T}\cdot\mathbf w$\\
	
	A $\mathbf v$ és $\mathbf w$ vektorok vektoriális szorzata:& ${\mathbf v} \times\mathbf w$\\
	
	Az a mátrix, amelyet bármely $\mathbf w$ vektorral & \\
	 \ \ \ \ jobbról megszorozva $(\mathbf v\times \mathbf w)$ értékét kapjuk: & $\mathbf v\times$\\
	
	Az $\mathbf M$ mátrix $\mathbf M = \mathbf U \cdot \mathbf S \cdot \mathbf V^\T$ SVD felbontása:& $[\mathbf U,\mathbf S, \mathbf V] = \svd{(\mathbf M)}$\\
	
	Az $\mathbf M$ mátrix inverze:& ${\mathbf M}^{-1}$\\
	
	Az $\mathbf M$ mátrix transzponáltja:& ${\mathbf M}^{\T}$\\
	
	Az $\mathbf M$ mátrix determinánsa:& $\det{(\mathbf M)}$\\
	
	Az $\mathbf M$ mátrix ún. Moore-Penrose pszeudoinverze :& ${\mathbf M}^{+}$\\
	
	Az $\mathbf M$ mátrix ún. csillapított pszeudoinverze, & \\
	 \ \ \ \ $\rho$ csillapítással :& ${\mathbf M}^{\rho +}$\\ %TODO: biztosan jó lesz ez magyarul?
	
	Az $\mathbf M$ mátrix nyoma (főátlóbeli elemeinek összege):& $\Trace{\mathbf M}$\\
	
	A $\mathbf t$ irány körüli $\varphi$ szögű elfordulást leíró transzformáció:& $\Rot{\mathbf t}{\varphi}$\\
	
	Az $x$, $y$ vagy $z$ tengely körüli $\varphi$ & \\
	\hspace{15mm} szögű elfordulást leíró transzformáció:& $\Rotx{\varphi}$, $\Roty{\varphi}$, $\Rotz{\varphi}$ \\
	
	A $\mathbf d$ mértékű elmozdulást leíró transzformáció:& $\Tran{\mathbf d}$ \\
	
	Az $x$, $y$ vagy $z$ irányú $d$ & \\
	\hspace{15mm} nagyságú elmozdulást leíró transzformáció:& $\Tranx{d}$, $\Trany{d}$, $\Tranz{d}$ \\
	
	Az aktuális példában egyértelmű, vagy indifferens méretű & \\
	\hspace{15mm} vektorok, mátrixok jelölése: & $\begin{bmatrix}
		c_1 \\ c_2 \\ \vdots
	\end{bmatrix}$, $\begin{bmatrix}
		m_{11} & m_{12} & \dots \\
		m_{21} & m_{22} & \dots \\
		\vdots & \vdots &
	\end{bmatrix}$ \\

	A $t$ érték a következő értékeket veheti fel: $a\leq t\leq b$ & $t\in[a,b]$
	
\end{tabular}
